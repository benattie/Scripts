%% start of file `jdoe_classic.tex'.
%% Copyright 2006 Xavier Danaux.
%
% This work may be distributed and/or modified under the
% conditions of the LaTeX Project Public License version 1.3c,
% available at http://www.latex-project.org/lppl/.


\documentclass[10pt]{moderncv}

% moderncv styles
\moderncvstyle{casual}       % optional argument are 'nocolor' (black & white cv) and 'roman' (for roman fonts, instead of sans serif fonts)
%\moderncvstyle{classic}       % idem

% character encoding
\usepackage[utf8]{inputenc}   % replace by the encoding you are using

% personal data (the given example is exhaustive; just give what you want)
\firstname{Emanuel Alejandro}
\familyname{Benatti}
\title{}
%\address{Av. Bustillo 9500}
\phone{(0294)-154-784179}

\email{emanuel.benatti@gmail.com}
%\extrainfo{\weblink{www.ctan.org}}

%%%%%%%%%%%%%%Descomenta esta linea si queres poner una foto tuya%%%%%%%%%%%%
%\photo[64pt]{cv.png} % also optional, and the optional argument is the height the picture must be resized to
%%%%Obviamente, reemplazá cv.png por tu foto, y acordate de poner la foto en la misma carpeta en la que esta el .tex
%%%%%%%%%%%%%%%%%%%%%%%%%%%%%%%%%%%%%%%%%%%%%%%%%%%%%%%%%%%%%%%%%%%%%%%%%%%%%
%\quote{Any intelligent fool can make things bigger, more complex, and more violent. It takes a touch of genius -- and a lot of courage -- to move in the opposite direction.}% also optional

%\renewcommand{\listsymbol}{{\fontencoding{U}\fontfamily{ding}\selectfont\tiny\symbol{'102}}} % define another symbol to be used in front of the list items

% the ConTeXt symbol
\def\ConTeXt{%
  C%
  \kern-.0333emo%
  \kern-.0333emn%
  \kern-.0667em\TeX%
  \kern-.0333emt}

% slanted small caps (only with roman family; the sans serif font doesn't exists :-()
%\usepackage{slantsc}
%\DeclareFontFamily{T1}{myfont}{}
%\DeclareFontShape{T1}{myfont}{m}{scsl}{ <-> cork-lmssqbo8}{}
%\usefont{T1}{myfont}{m}{scsl}Testing the font

% command and color used in this document, independently from moderncv 
\definecolor{see}{rgb}{0.5,0.5,0.5}% for web links
\newcommand{\up}[1]{\ensuremath{^\textrm{\scriptsize#1}}}% for text subscripts

%----------------------------------------------------------------------------------
%            content
%----------------------------------------------------------------------------------
\begin{document}
\maketitle
%\makequote


\section{Datos personales}
\cvitem{Fecha de}{}
\cvitem{nacimiento:}{30 de Diciembre de 1988.}
\cvitem{Lugar de}{}
\cvitem{nacimiento:}{Rosario, Santa Fe, Argentina.}
\cvitem{DNI:}{34.288.650}
\cvitem{CUIL:}{20-34288650-8}
\cvitem{Dirección:}{Juan Martín de Pueyrredón 3463, Rosario, Santa Fe}
\cvitem{Código postal:}{2000}
\cvitem{Teléfono:}{(0294)-154-784179}
\cvitem{Correo}{}
\cvitem{electrónico:}{emanuel.benatti@gmail.com}
\section{Formación}
\cventry{2012}{Maestría en Ciencias Físicas}{Especialización en física tec\-no\-ló\-gi\-ca.}{Universidad de Cuyo - Instituto Balseiro}{}{}
\cventry{2009-2011}{Licenciado en física}{Universidad de Cuyo - Instituto Balseiro}{}{}{}
\cvitem{Promedio}{}
\cvitem{general:}{8,17 (Referencia: promedio histórico 7,39)}

\section{Tesis de grado}
\cvitem{Título:}{\textbf{Diseño y construcción de un detector de neutrones utilizando el superconductor MgB$_2$}}{}
\cvitem{Director:}{Dr. Mariano Gómez Berisso}
\cvitem{Co-Director:}{Dr. Julio Guimpel}

\section{Tesis de maestría}
\cvitem{Título:}{\textbf{Films de MgB$_2$: posibilidad de uso como detector de neutrones}}{}
\cvitem{Director:}{Dr. Mariano Gómez Berisso}
\cvitem{Co-Director:}{Dr. Julio Guimpel}

\section{Congresos}	
\cvitem{}{Asistencia a la Segunda Reunión Conjunta SUF-AFA. XII Reunión de la SUF - 96\textordfeminine\,Reunión Nacional de la Asociación Física Argentina, Montevideo Uruguay, 20 y 23 de Septiembre de 2011.}
\cvitem{}{Presentación de Póster en la Segunda Reunión Conjunta SUF-AFA. XII Reunión de la SUF - 96\textordfeminine\,Reunión Nacional de la Asociación Física Argentina:}
\cvlistitem{Estudio de la estructura de vórtices en los superconductores de alta temperatura crítica FeSe$_{1-x}$Te$_{x}$ y Bi$_{2}$Sr$_{2}$CaCu$_{2}$O$_{8}$ \textbf{Benatti\,E.}, Fasano Y., Nieva G., Gaspar Franco D.}

\cvitem{}{Asistencia a la 97\textordfeminine\,Reunión Nacional de la Asociación Física Argentina, Carlos Paz, Córdoba, 25 y 28 de Septiembre de 2012.}
\cvitem{}{Presentación de Póster en la 97\textordfeminine\,Reunión Nacional de la Asociación Física Argentina:}
\cvlistitem{Diseño y construcción de un detector de neutrones utilizando el superconductor MgB$_2$ \textbf{Benatti E.}, Gómez Berisso M., Guimpel\,J.}

\section{Materias de Postgrado}
\cvlistitem{Fenomenología de la materia condensada}
\cvlistitem{Interacción de neutrones con la materia condensada}
\cvlistitem{Tópicos de física computacional}
\cvlistitem{Magnetismo}
\cvlistitem{Superconductividad}
\cvlistitem{Caracterización de materiales}
\cvlistitem{Economía y gestión de proyectos}
\cvlistitem{Introducción a los microsistemas (MEMS) y la microfabricación}

%\newpage
\section{Becas}
\cvitem{2012}{Beca para realizar la Maestría en Ciencias Físicas del Instituto Balseiro, otorgada por la fundación YPF.}
\cvitem{2009-2011}{Beca para realizar la carrera de Licenciatura en Física en el Instituto Balseiro, otorgada por la fundación YPF.}

\section{Idiomas}
\cvlanguage{Español}{Nativo}{}
\cvlanguage{Ingles}{Nivel Intermedio}{}
\closesection{}
%\pagebreak{}

\section{Conocimientos de computación y programación}
\cvitem{OS}{Linux, Windows}{}{}
\cvitem{Lenguajes}{C/C++, Labview, LaTex, Visual Basic}{}{}
\cvitem{Programas}{gnuplot, vim, coreutils, Open Office, Layout Editor, Comsol Multiphysics, AutoCAD, Origin}{}{}

\section{Otros antecedentes}
\cvlistitem{Expositor del Stand del Laboratorio de Bajas Temperaturas en la Muestra CAB-IB en el año 2012.}
\cvlistitem{Colaboración en la Semana de Ciencia y Tecnología realizada en el año 2012.}
\cvlistitem{Tesorero del Centro de Estudiantes del Instituto Balseiro en los años 2010-2011.}
\cvlistitem{Presidente del Centro de Estudiantes del Instituto Balseiro en los años 2011-2012.}
\end{document}


%% end of file `jdoe_classic.tex'.
