% awk
%Extrer las columnas 1 y 2 de Al-AR-M-H-tex_00001.spr_trs.csv (archivo de valores separados por coma) y redirigir la salida desde pantalla al archivo dif000.dat (los datos en dif000.dat aparecen separados por un espacio)
awk -F, '{print $1 " " $2}' Al-AR-M-H-tex_00001.spr_trs.csv > dif000.dat
% Eliminar la tercer columna
awk '!($7=" ")' input > output
$
------------------------------------------------------------------------------------
% coreutils en general
% sed
%extraer el primer caracter de la primer linea de tmp y mandarlo a tmp2
sed '1s/^.//' tmp > tmp2
%elimina los primeros 2 caracteres
sed -i 's/^..//' print_pole_figure.py
%elimina los primeros 3 caracteres
sed -i 's/^...//' print_pole_figure.py
%crea un directorio y todos los directorios superiores indicados en el path
mkdir -p /ruta/al/directorio
------------------------------------------------------------------------------------
% cut
%Extraer columnas 1, 2 y 4 de Al-AR-M-H-tex_00001.spr_trs.csv (archivo de valores separados por coma) y redirigir la salida desde pantalla al archivo pp.dat. Los valores aparecen igual que en el archivo original
cut -d',' -f1,2,4 Al-AR-M-H-tex_00001.spr_trs.csv > pp.dat
%Extraer columnas 1 a la 4 de Al-AR-M-H-tex_00001.spr_trs.csv (archivo de valores separados por coma) y redirigir la salida desde pantalla al archivo pp.dat. Los valores aparecen igual que en el archivo original
cut -d',' -f1-4 Al-AR-M-H-tex_00001.spr_trs.csv > pp.dat
%elimina los primeros n - 1 caracteres
cut -c n- fit2peak-index.m > fit2peak-index.m
------------------------------------------------------------------------------------
% Git
%ver como esta todo
git status
%ver historia de los commits
git log
%agregar archivo creado/modificado
git add file
%agregar todo lo cambiado
git add .
%renombrar archivo
git mv old_file new_file
%borrar archivo
git rm file
%borrar directorio
git rm -r folder
%crear un commit con los archivos modificados
git commit -m "<comentario>"
%agregar archivos modificados y crear el commit al mismo tiempo
git commit -a
%subir cambios a repositorio remoto
git push <url> branch
%obtener cambios desde repositorio remoto
git pull <url> branch
%agregar repositorio remoto
git remote add repo_name <url>
%checkear por cambios sin modificar el repositorio local
git fetch repo_name
git diff master repo_name/master
%combinar los repositorios una vez checkeadas las diferencias
git merge origin/master
%clonar repositorio remoto
git clone username@host:/path/to/repository
%deshacer cambios en el archivo file (que no fue agregado con git add)
git checkout -- file
%deshacer todo a la ultima version que figura en el repositorio remoto
git fetch origin o git fetch <url> branch
git reset --hard origin/master
%crear y cambiar a brach feature_x
git checkout -b feature_x
%cambiar a branch x
git checkout x
%renombrar branch
git branch -m old_branch new_branch
%borrar branch feature_x
git branch -d feature_x
%combinar branch feature_x a rama activa
git merge feature_x
%ver todas las ramas creadas
git branch
%gui
git gui
------------------------------------------------------------------------------------
% vim
%reemplazar todas las ocurrencias de foo por bar
:%s/foo/bar/g
%reemplazar todas las ocurrencias de foo por bar pidiendo confirmacion
:%s/foo/bar/gc
%borrar hasta el fin de linea
D
%borrar hasta la proxima palabra
dw
------------------------------------------------------------------------------------
% gnuplot
%graficar datos ingresados a mano
plot '-'
------------------------------------------------------------------------------------
% diff
%salida en columnas mostrando solo en la izquierda las filas en comun
diff -y --left-column a001_PF_1_m.dat a001_PF_1_m.dat.orig
%idem anterior omitiendo diferencias por tabs
diff -yE --left-column a001_PF_1_m.dat a001_PF_1_m.dat.orig
%idem anterior omitiendo diferencias por espacios
diff -yb --left-column a001_PF_1_m.dat a001_PF_1_m.dat.orig
------------------------------------------------------------------------------------
% creacion de iso
% crear imagen iso de un CD/DVD
dd if=/dev/hdc of=archivo.iso
%crear iso a partir de un directorio
mkisofs -o /tmp/cd.iso /tmp/directory/
------------------------------------------------------------------------------------
% rename
rename 's/search/replace/g'
------------------------------------------------------------------------------------
% convert
convert a.pdf a.png
% jugando con la calidad
convert -quality 00 a.pdf a.png
convert -quality 05 a.pdf a.png
------------------------------------------------------------------------------------
% Cosas varias
% ruta a una carpeta compartida por samba
/run/user/$(id -u)/gvfs
$
% Comprimir una carpeta usando 7zip
7z a -t7z -mx=9 file7z.7z dir
% Comprimir una carpeta usando 7zip con contraseña abre
7z a -t7z -mx=9 -pabre file7z.7z dir
% Copiar un directorio al cluster
rsync -avz /origin/path/ ebenatti.ifir@piluso.rosario-conicet.gov.ar:/destination/path/
